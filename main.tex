\documentclass[11pt,a4paper]{article}
\usepackage[usenames,dvipsnames]{color}
\usepackage{iccv}
\usepackage{times}
\usepackage{epsfig}
\usepackage{graphicx}
\usepackage{amsmath}
\usepackage{amssymb}
\usepackage[brazil]{babel}
\usepackage[OT1]{fontenc}
\usepackage[utf8]{inputenc}
\usepackage[a4paper,
hmargin={1.5cm,1.5cm},
vmargin={2cm,2.5cm},
footskip=5mm]{geometry}

\definecolor{azullink}{rgb}{0,0,1}

\usepackage[pagebackref=true,breaklinks=true,letterpaper=false,colorlinks,bookmarks=false]{hyperref}
\hypersetup{
 colorlinks,
 citecolor=azullink,
 linkcolor=Red,
 urlcolor=Blue}

\begin{document}


\title{O modelo Spotify: uma análise exploratória sobre a adaptabilidade dos métodos ágeis em grandes empresas\\ \smallskip
\small{ PROPOSTA DE TRABALHO}}

\author{ \bf Alécio Dalprá\\
		\tt aleciodalpra@gmail.com \\
		Curso de Especialização em Tecnologias Aplicadas a Sistemas de Informação com Métodos Ágeis \\
		Centro Universitário Ritter Dos Reis - UNIRITTER 
}

\maketitle
\thispagestyle{empty}




\section{Relevância do trabalho} \label{sec:intro}

O Spotify \footnote{{https://www.spotify.com/br/}}, empresa sueca de streaming de música, é sem sombra de dúvidas, uma empresa de sucesso, está na moda e entre as startups mais invejadas do mercado, não restrito apenas ao mercado musical. Seu negócio está embasado e depende totalmente de TI (Tecnologia da Informação).

Nos dias de hoje são raras as empresas que podem prescindir da TI e continuar sendo competitivas no seu ramo de atuação. Essa necessidade faz com que as empresas optem por diferentes formas de gerenciamento de seus projetos, desde simples processos ad-hoc a controles complexos, como um diagrama de Gantt \cite{Lourenco:Dissertacao}.

O sucesso do Spotify e a constante ascensão do desenvolvimento de software baseado em princípios ágeis publicados no Manifesto Ágil \footnote{http://agilemanifesto.org/} ainda em 2001, fez com que muitas empresas literalmente se jogassem nesse modelo de trabalho esperando que todos os seus problemas fossem resolvidos de uma forma mágica a partir da simples adoção de um kit de práticas dos modelos ágeis mais conhecidos, como Lean Development \cite{MaryTomPoppendieck}, Scrum \cite{Schwaber}, Extreme Programing (XP) \cite{BeckFowler}, Kanban \cite{Rasmusson}. É o sonho de se tornar o Spotify do dia pra noite.

As empresas, no entanto, estão inseridas em contextos distintos. Seja um contexto de negócio como um mercado específico ou, para aquelas que estão no mesmo setor, há questões organizacionais próprias, tais como a hierarquia, tamanho da empresa, tempo no mercado, particularidades de produtos e clientes, arquitetura tecnológica, dentre outras. O modelo Spotify também possui suas particularidades de acordo com as necessidades e contexto do próprio Spotify. Algumas consultorias e consultores independentes têm oferecido serviços e treinamentos com discursos de maior lucratividade e resposta ao mercado se adotarem o método ágil A ou B (dentre eles o modelo Spotify) sem considerar os detalhes do contexto. Em muitos casos a iniciativa falha e a culpa é atribuída ao método.

Atualmente, encontra-se na bibliografia trabalhos que relatam a adoção de métodos ágeis por algumas empresas como \cite{EstudoCasoSERPRO}, \cite{MitigacaoDificuldadesSERPRO},  \cite{melo2010adoccao}, \cite{mauricio2016utilizaccao}. No entanto, estes textos abordam um método ou situações específicas das empresas analisadas. Nenhum dos textos cita a adoção do modelo Spotify. Os que mencionam o Spotify referem-se a aspectos específicos da plataforma ou dos usuários, como: \cite{kreitz2010spotify}, \cite{zhang2013understanding}, \cite{goldmann2011measurements}.

A partir desta análise, observa-se uma carência de trabalhos exploratórios apresentando a viabilidade para adoção do modelo Spotify bem como resultados obtidos a partir da implantação das respectivas técnicas de trabalho em diferentes contextos.


\section{Objetivos}\label{sec:objetivos}

\subsection{Objetivo geral}

Realizar uma análise exploratória sobre o modelo Spotify de trabalho e a adoção deste modelo por empresas de diferentes setores.

\subsection{Objetivos específicos}

\begin{itemize}
  \item Analisar o contexto do Spotify e necessidades que levaram a empresa a construir o referido modelo;
  \item Identificar os processos, atividades e estrutura e modelo de organização proposto;
  \item Avaliar a aplicabilidade e não aplicabilidade do modelo Spotify em outros contextos e empresas;
\end{itemize}

\section{Solução proposta	}

Este trabalho visa avaliar o contexto de trabalho do Spotify e as necessidades que o levaram a adotar determinadas práticas no dia a dia e a desenvolver um modelo de trabalho que proporcionou uma melhor resposta aos desafios impostos pelo mercado, bem como analisar se o modelo possui alguma limitação e oportunidade de melhoria na visão da própria empresa.

Também é propósito identificar, no que foi proposto e desenvolvido pelo Spotify, padrões que podem ser adotados por outras empresas de diferentes setores do mercado. Tão importante quanto identificar o que se aplica também é o desafio de identificar o que não se aplica e a elaboração de alternativas para um processo de transformação ágil conforme o contexto onde a empresa está inserida.

As análises serão realizadas a partir de materiais publicados pelo Spotify e disponíveis na internet, bem como o contexto de empresas brasileiras para comparação da aplicabilidade do modelo.

\section{Cronograma de desenvolvimento}\label{sec:cronograma}

A Tabela \ref{tab:cronograma} apresenta o cronograma de desenvolvimento do trabalho conforme a numeração das atividades abaixo:
\begin{enumerate}
	\item Investigação e identificação do histórico do Spotify	e coleta de dados sobre processos e organização estrutural da empresa;
	\item Estudo sobre as técnicas utilizadas e seus respectivos contextos; 
	\item Seleção de algumas empresas brasileiras para amostra de comparação;
	\item Avaliação da aplicabilidade do modelo do Spotify nas respectivas empresas;
	\item Elaboração do artigo;
	\item Apresentação do trabalho.
\end{enumerate}

\begin{table}[ht]
\begin{center}
	\caption{Cronograma de atividades. \label{tab:cronograma}}
		\begin{tabular}{|c|c|c|c|c|c|c|}
			\hline			
			 \bf Atividade & \bf 2/18 & \bf 3/18 & \bf 4/18 & \bf 5/18 & \bf 6/18 & \bf 8/18  \\	\hline \hline
					 1 & x & x &   &   &   &   \\ \hline
					 2 &   & x & x & x &   &   \\ \hline
					 3 & x & x & x &   &   &   \\ \hline
					 4 & x & x & x & x &   &   \\ \hline
					 5 &   &   &   & x & x &   \\ \hline
					 6 &   &   & x & x & x &   \\ \hline
		\end{tabular}
		\end{center}		
\end{table}

\renewcommand\refname{Referências}
{\small
\bibliographystyle{ieee}
\bibliography{MinhasReferencias}
}

\end{document}
